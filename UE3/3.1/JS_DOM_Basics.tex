%%
%% Author: thompson
%% 02.11.17
%%

% Preamble
\documentclass[11pt]{article}

% Packages
\usepackage{a4wide}
\usepackage{verbatim}       % Package for "Sourcecode"
\usepackage{scrextend}
\usepackage{enumerate}


% FIXME: Linkreferences by footnotes are buggy
% Document
\begin{document}
    \section{JS and DOM Basics}
    \begin{enumerate}[\thesection .1]
        \item What is the purpose of Javascript and how does it complement HTML and CSS?\footnote[1]{Src.: $https://www.w3schools.com/js/js\_whereto.asp $}
        Javascript gilt als Programmiersprache, ähnlich wie Java, Rust, php, ....
        Javascript wird mithilfe des $<$script$>$-Tags in ein HTML-Dokument eingebunden.
        Ein externales sowie internales Einbinden von Javascript ist zu jeder Zeit möglich, egal ob es sich dabei um Sourcecode
        innerhalb des Head oder des Body befindet. (Siehe demoHTML.html)

        \begin{verbatim}
        <!DOCTYPE html>
          <html>
            <head>
            <script src="demoJS.js"></script>
            </head>

            <body>
              <h1>A Web Page</h1>
              <p id="demo">A Paragraph</p>
              <button type="button" onclick="myFunction()">Try it</button>

            </body>
          </html>
        \end{verbatim}
        Mithilfe von JS lassen sich Funktionen effizient schreiben.
        Es gilt dabei zu beachten, dass es zu den guten Praktiken gehört, JS-Code von HTML und CSS zu trennen.
        Dementsprechend ist ein externales Einbinden anstrebbar.
        \begin{enumerate}
            \item[$\diamond$] Vorteile und Nachteile von Internen Code
            \begin{enumerate}
                \item[+] Schnelles Testing \& Debuggen
                \item[-] Lange Quellcodes führen zu unübersichtlichen Code
                \item[-] Langer Wartungsaufwand bei dupliziertem Quellcode
            \end{enumerate}

            \item[$\diamond$] Vorteile und Nachteile von Externem Code
            \begin{enumerate}
                \item[+] Übersichtlicher Code
                \item[+] Kurzer Wartungsaufwand
                \item[+] Zwischengespeicherter JS-Code beschleunigt Ladezeiten
            \end{enumerate}
        \end{enumerate}

        \item What kind of typing is provided by Javascript? What are the risks?\footnote[2]{Src.: $https://www.w3schools.com/js/js\_type\_conversion.asp$}
        In Javascript unterscheidet man zwischen folgenden
        \begin{addmargin}[1em]{1em}
            Typen:
            \begin{enumerate}[$\circ$]
                \item String
                \item Number
                \item Boolean
                \item Object
                \item Function
                \item null
                \item undefined
            \end{enumerate}

            Objekten:
            \begin{enumerate}[$\circ$]
                \item Object
                \item Date
                \item Array
            \end{enumerate}




        \end{addmargin}



        \item What is DOM?

    \end{enumerate}




\end{document}