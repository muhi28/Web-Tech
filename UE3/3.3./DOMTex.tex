%%
%% Author: thompson
%% 03.11.17
%%

% Preamble
\documentclass[11pt]{article}

% Packages
\usepackage{a4wide}
\usepackage[ngerman]{babel}
\usepackage[utf8]{inputenc}

\usepackage{enumerate}
\usepackage{verbatim}
\usepackage{scrextend}

% Document
\begin{document}
    \section{Explanation of DOM-Tree}
    \emph{Siehe Beiligendes .js und .html}
    \begin{verbatim}
        var link = document.getElementsByTagName("a")[2];
    \end{verbatim}
    Liest Elemente des .html ein welches das .js einbindet, dessen Element-Tag $<$a$>$ ist.
    Eingelesene Daten werden in einem Array abgelegt, welches als 'link' definiert wird.

    Für unser .html, welches so aussieht:
    \begin{verbatim}
        <a href="http://www.orf.at/" target="_new">Link zum ORF</a><br>
        <a href="http://www.sms.at" target="_new">Link zu sms.at</a><br>
        <a href="http://www.aau.at" target="_new">Link zur AAU Seite</a>
    \end{verbatim}
    .. werden die Link-Adressen http://* in das oben definierte Array eingelesen.

    \begin{enumerate}[1]
        \item Block:
        \begin{verbatim}
            alert (link.attributes["href"].nodeValue);
            alert (link.getAttributeNode("href").nodeValue);
            alert (link.href);
        \end{verbatim}
        Die erste Zeile\\
        Die zweite Zeile\\
        Die dritte Zeile\\

        \item Block:
        \begin{verbatim}
            alert(link.firstChild.data);
        \end{verbatim}
            Stuff about it
        \item Block:
        \begin{verbatim}
            alert (link.attributes[0].nodeType);
            alert (link.firstChild.nodeType);
        \end{verbatim}
        Stuff about that
        \item Block:
        \begin{verbatim}
            alert(link.parentNode.nodeName);
            alert(link.lastChild.parentNode.nodeName);
        \end{verbatim}
        stuff about that too
        \item Block:
        \begin{verbatim}
            alert(link.attributes.length);
        \end{verbatim}
        And stuff about that
    \end{enumerate}



\end{document}