%%
%% Author: thompson
%% 03.11.17
%%

% Preamble
\documentclass[11pt]{article}

% Packages
\usepackage{a4wide}
\usepackage[ngerman]{babel}
\usepackage[utf8]{inputenc}

\usepackage{enumerate}
\usepackage{verbatim}
\usepackage{scrextend}
\usepackage{mathtools}

% Document
\begin{document}
    \section{Explanation of DOM-Tree}
    \emph{Siehe Beiligendes .js und .html}\\
    Für unser .html, welches so aussieht:
    \begin{verbatim}
        <a href="http://www.orf.at/" target="_new">Link zum ORF</a><br>
        <a href="http://www.sms.at" target="_new">Link zu sms.at</a><br>
        <a href="http://www.aau.at" target="_new">Link zur AAU Seite</a>
        <script src="DOMTree.js"></script>
    \end{verbatim}
    .. wurde ein .js definiert, welches in das .html eingebunden wird.

    Das .js beginnt mit:
    \begin{verbatim}
        var link = document.getElementsByTagName("a")[2];
    \end{verbatim}
    .. einer Schaffung und Wertzuweisung der Variable 'link'.
    'link' wird dem 3. Element des Arrays gleichgesetzt, welches rechtsseitiger definiert wird.\\
    e.g.: var link = urls[2] -- $<a> this.data </a>$
    \begin{enumerate}[$\diamond$]
        \item Block 1
        \begin{verbatim}
        alert (link.attributes["href"].nodeValue);
        alert (link.getAttributeNode("href").nodeValue);
        alert (link.href);
        \end{verbatim}
        Alle 3 Zeilen machen dasselbe: Sie geben das Arugment des href-Wertes wieder.
        link.attributes["href"].nodeValue greift direkt auf den Link zu, welcher in href="..." steht.
        link.getAttributeNode("href").nodeValue greift über die Attributsnode "href" zu und schnappt sich dessen Wert: Der Text von href="x".
        link.href ist der spezifischte Zugriff, direkt auf href des Links.

        \item Block 2
        \begin{verbatim}
            alert(link.firstChild.data);
        \end{verbatim}
        Diese Zeile gibt die Kind-Data wieder, in unserem Link ist das der inbeschriebene Text:
        'Link zur AAU Seite'

        \item Block 3
        \begin{verbatim}
            alert (link.attributes[0].nodeType);
            alert (link.firstChild.nodeType);
        \end{verbatim}
        Die erste Zeile dieses Blocks beschreibt den NodeType, welcher 2 ist.
        '2' steht hier ledigich für ein Element, dessen Attribute per se beschrieben werden mit '2'.
        Gemäß Standart (DOM-L4), gibt diese Abfrage immer 2 zurück.\footnote[1]{$https://developer.mozilla.org/de/docs/Web/API/Node/nodeType$}
        Zweitere gibt den Typ des ersten Kindes wieder. Das Kind unseres $<a></a>$ ist der darinliegende Text.
        Die Attributszahl von Text ist 3.

        \item Block 4
        \begin{verbatim}
            alert(link.parentNode.nodeName);
            alert(link.lastChild.parentNode.nodeName);
        \end{verbatim}
        Die erste Zeile beschreibt den Namen des Parent-Node. Der Parent-Node unseres $<a></a>$ ist der $<body></body>$,
        dementsprechend liefert dies BODY zurück.
        ParentNode.nodeName gibt den Namen der Parent-Node des letzten Kindsknoten wieder.
        Hier 'A'.

        \item Block 5:
        \begin{verbatim}
            alert(link.attributes.length);
        \end{verbatim}
        Diese Zeile gibt lediglich die Anzahl der Attribute wieder, die im Link spezifiziert wurden.
        mit href und target bestehen hier 2 Attribute, wodurch der Wert 2 ist.

    \end{enumerate}
\end{document}